%%%%%%%%%%%%%%%%%%%%%%%%%%%%%%%%%%%%%%%%%
%
% CMPT 432
% Spring 2018
% Lab Three
%
%%%%%%%%%%%%%%%%%%%%%%%%%%%%%%%%%%%%%%%%%

%%%%%%%%%%%%%%%%%%%%%%%%%%%%%%%%%%%%%%%%%
% Short Sectioned Assignment
% LaTeX Template
% Version 1.0 (5/5/12)
%
% This template has been downloaded from: http://www.LaTeXTemplates.com
% Original author: % Frits Wenneker (http://www.howtotex.com)
% License: CC BY-NC-SA 3.0 (http://creativecommons.org/licenses/by-nc-sa/3.0/)
% Modified by Alan G. Labosueur  - alan@labouseur.com
%
%%%%%%%%%%%%%%%%%%%%%%%%%%%%%%%%%%%%%%%%%

%----------------------------------------------------------------------------------------
%	PACKAGES AND OTHER DOCUMENT CONFIGURATIONS
%----------------------------------------------------------------------------------------

\documentclass[letterpaper, 10pt,DIV=13]{scrartcl} 

\usepackage[T1]{fontenc} % Use 8-bit encoding that has 256 glyphs
\usepackage[english]{babel} % English language/hyphenation
\usepackage{amsmath,amsfonts,amsthm,xfrac} % Math packages
\usepackage{sectsty} % Allows customizing section commands
\usepackage{graphicx}
\usepackage[lined,linesnumbered,commentsnumbered]{algorithm2e}
\usepackage{listings}
% Added package for listing by letter
\usepackage{enumitem}
%
\usepackage{parskip}
\usepackage{lastpage}

\allsectionsfont{\normalfont\scshape} % Make all section titles in default font and small caps.

\usepackage{fancyhdr} % Custom headers and footers
\pagestyle{fancyplain} % Makes all pages in the document conform to the custom headers and footers

\fancyhead{} % No page header - if you want one, create it in the same way as the footers below
\fancyfoot[L]{} % Empty left footer
\fancyfoot[C]{} % Empty center footer
\fancyfoot[R]{page \thepage\ of \pageref{LastPage}} % Page numbering for right footer

\renewcommand{\headrulewidth}{0pt} % Remove header underlines
\renewcommand{\footrulewidth}{0pt} % Remove footer underlines
\setlength{\headheight}{13.6pt} % Customize the height of the header

\numberwithin{equation}{section} % Number equations within sections (i.e. 1.1, 1.2, 2.1, 2.2 instead of 1, 2, 3, 4)
\numberwithin{figure}{section} % Number figures within sections (i.e. 1.1, 1.2, 2.1, 2.2 instead of 1, 2, 3, 4)
\numberwithin{table}{section} % Number tables within sections (i.e. 1.1, 1.2, 2.1, 2.2 instead of 1, 2, 3, 4)

\setlength\parindent{0pt} % Removes all indentation from paragraphs.

\binoppenalty=3000
\relpenalty=3000

\lstset { % MAZ Styling for writing out grammar
  basicstyle = \itshape,
  xleftmargin = 3em,
  literate = {->}{$\rightarrow$}{2} {-->}{$\Rightarrow$}{2}
}

%----------------------------------------------------------------------------------------
%	TITLE SECTION
%----------------------------------------------------------------------------------------

\newcommand{\horrule}[1]{\rule{\linewidth}{#1}} % Create horizontal rule command with 1 argument of height

\title{	
   \normalfont \normalsize
   \textsc{CMPT 432 - Spring 2018 - Dr. Labouseur} \\[10pt] % Header stuff.
   \horrule{0.5pt} \\[0.25cm] 	% Top horizontal rule
   \huge Lab Three  \\     	    % Assignment title
   \horrule{0.5pt} \\[0.25cm] 	% Bottom horizontal rule
}

\author{Marcus A. Zimmermann \\ \normalsize Marcus.Zimmermann1@Marist.edu}

\date{\normalsize\today} 	% Today's date.

\begin{document}
\maketitle % Print the title

%----------------------------------------------------------------------------------------
%   start CRAFTING A COMPILER EXERCISES
%----------------------------------------------------------------------------------------
\section*{Crafting a Compiler}
\subsection*{Exercise 4.7}
Grammar for infix expressions:
\begin{lstlisting}
Start -> E $
E     -> T plus E
       | T
T     -> T times F
       | F
F     -> ( E )
       | num
\end{lstlisting}

\begin{enumerate}[label=\textbf{\Alph*}]
\item Show the leftmost derivation of the following string.\\ \\
num plus num times num plus num \$ \\
\begin{lstlisting}
Start --> E $
      --> T plus E $
      --> F plus E $
      --> num plus E $
      --> num plus T plus E $
      --> num plus T times F plus E $
      --> num plus F times F plus E $
      --> num plus num times F plus E $
      --> num plus num times num plus E $
      --> num plus num times num plus T $
      --> num plus num times num plus F $
      --> num plus num times num plus num $
\end{lstlisting}

\pagebreak

\item Show the rightmost derivation of the following string.\\ \\
num times num plus num times num \$ \\
\begin{lstlisting}
Start --> E $
      --> T plus E $
      --> T plus T $
      --> T plus T times F $
      --> T plus T times num $
      --> T plus F times num $
      --> T plus num times num $
      --> T times F plus num times num $
      --> T times num plus num times num $
      --> F times num plus num times num $
      --> num times num plus num times num $
\end{lstlisting}

\item Describe how this grammar structures expressions, in terms of the precedence and left- or right- associativity of operators.\\ \\
This grammar ensures that "times" has a higher precedence than "plus." Different nonterminals are used for each precedence level. By placing the rewrite rule for "times" lower in the grammar, it ends up lower in the tree.
\end{enumerate}

\subsection*{Exercise 5.2}
\begin{enumerate}[label=\Alph*]
\item ... \\
\end{enumerate}

%----------------------------------------------------------------------------------------
%   end CRAFTING A COMPILER EXERCISES
%----------------------------------------------------------------------------------------
\pagebreak

%----------------------------------------------------------------------------------------
%   start DRAGON BOOK EXERCISES
%----------------------------------------------------------------------------------------
\section*{Compilers: Principles, Techniques, and Tools}
\subsection*{Exercise 4.2.1}

%----------------------------------------------------------------------------------------
%   end DRAGON BOOK EXERCISES
%----------------------------------------------------------------------------------------
\pagebreak

%----------------------------------------------------------------------------------------
%   REFERENCES
%----------------------------------------------------------------------------------------
% The following two commands are all you need in the initial runs of your .tex file to
% produce the bibliography for the citations in your paper.
\bibliographystyle{abbrv}
\bibliography{lab} 
% You must have a proper ".bib" file and remember to run:
% latex bibtex latex latex
% to resolve all references.

\end{document}
