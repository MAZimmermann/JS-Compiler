%%%%%%%%%%%%%%%%%%%%%%%%%%%%%%%%%%%%%%%%%
%
% CMPT 432
% Spring 2018
% Lab Nine
%
%%%%%%%%%%%%%%%%%%%%%%%%%%%%%%%%%%%%%%%%%

%%%%%%%%%%%%%%%%%%%%%%%%%%%%%%%%%%%%%%%%%
% Short Sectioned Assignment
% LaTeX Template
% Version 1.0 (5/5/12)
%
% This template has been downloaded from: http://www.LaTeXTemplates.com
% Original author: % Frits Wenneker (http://www.howtotex.com)
% License: CC BY-NC-SA 3.0 (http://creativecommons.org/licenses/by-nc-sa/3.0/)
% Modified by Alan G. Labosueur  - alan@labouseur.com
%
%%%%%%%%%%%%%%%%%%%%%%%%%%%%%%%%%%%%%%%%%

%----------------------------------------------------------------------------------------
%	PACKAGES AND OTHER DOCUMENT CONFIGURATIONS
%----------------------------------------------------------------------------------------

\documentclass[letterpaper, 10pt,DIV=13]{scrartcl} 

\usepackage[T1]{fontenc} % Use 8-bit encoding that has 256 glyphs
\usepackage[english]{babel} % English language/hyphenation
\usepackage{amsmath,amsfonts,amsthm,xfrac} % Math packages
\usepackage{sectsty} % Allows customizing section commands
\usepackage{graphicx}
\usepackage[lined,linesnumbered,commentsnumbered]{algorithm2e}
\usepackage{listings}
% Added package for listing by letter
\usepackage{enumitem}
%
\usepackage{parskip}
\usepackage{lastpage}

\allsectionsfont{\normalfont\scshape} % Make all section titles in default font and small caps.

\usepackage{fancyhdr} % Custom headers and footers
\pagestyle{fancyplain} % Makes all pages in the document conform to the custom headers and footers

\fancyhead{} % No page header - if you want one, create it in the same way as the footers below
\fancyfoot[L]{} % Empty left footer
\fancyfoot[C]{} % Empty center footer
\fancyfoot[R]{page \thepage\ of \pageref{LastPage}} % Page numbering for right footer

\renewcommand{\headrulewidth}{0pt} % Remove header underlines
\renewcommand{\footrulewidth}{0pt} % Remove footer underlines
\setlength{\headheight}{13.6pt} % Customize the height of the header

\numberwithin{equation}{section} % Number equations within sections (i.e. 1.1, 1.2, 2.1, 2.2 instead of 1, 2, 3, 4)
\numberwithin{figure}{section} % Number figures within sections (i.e. 1.1, 1.2, 2.1, 2.2 instead of 1, 2, 3, 4)
\numberwithin{table}{section} % Number tables within sections (i.e. 1.1, 1.2, 2.1, 2.2 instead of 1, 2, 3, 4)

\setlength\parindent{0pt} % Removes all indentation from paragraphs.

\binoppenalty=3000
\relpenalty=3000

\lstset { % MAZ Styling for writing out grammar
  basicstyle = \itshape,
  xleftmargin = 3em,
  literate = {->}{$\rightarrow$}{2} {-->}{$\Rightarrow$}{2},
  escapeinside={(*@}{@*)},
}

%----------------------------------------------------------------------------------------
%	TITLE SECTION
%----------------------------------------------------------------------------------------

\newcommand{\horrule}[1]{\rule{\linewidth}{#1}} % Create horizontal rule command with 1 argument of height

\title{	
   \normalfont \normalsize
   \textsc{CMPT 432 - Spring 2018 - Dr. Labouseur} \\[10pt] % Header stuff.
   \horrule{0.5pt} \\[0.25cm] 	% Top horizontal rule
   \huge Lab Nine  \\     	    % Assignment title
   \horrule{0.5pt} \\[0.25cm] 	% Bottom horizontal rule
}

\author{Marcus A. Zimmermann \\ \normalsize Marcus.Zimmermann1@Marist.edu}

\date{\normalsize\today} 	% Today's date.

\begin{document}
\maketitle % Print the title

%----------------------------------------------------------------------------------------
%   start CRAFTING A COMPILER EXERCISES
%----------------------------------------------------------------------------------------
\section*{Crafting a Compiler}
\subsection*{Exercise 5.5}
Transform the following grammar into LL(1) form using the techniques presented in Section 5.5:
\begin{lstlisting}
DeclList       -> DeclList ; Decl
		| Decl
Decl	       -> IdList : Type
IdList	       -> IdList , id
		| id
Type	       -> ScalarType
		| array ( ScalarTypeList ) of Type
ScalarType     -> id
		| Bound .. Bound
Bound 	       -> Sign intconstant
		| id
Sign 	       -> +
		| -
		|(*@ $\lambda$ @*)
ScalarTypeList -> ScalarTypeList, ScalarType
		| ScalarType
\end{lstlisting}

\pagebreak

\subsection*{Exercise 5.5 Continued}
Above grammar after eliminating left recursion:
\begin{lstlisting}
DeclList       -> A B
A              -> Decl
B              -> ; Decl B
                |(*@ $\lambda$ @*)
Decl           -> IdList : Type
IdList         -> C D
C              -> id
D              -> , id D
                |(*@ $\lambda$ @*)
Type           -> ScalarType
                | array ( ScalarTypeList ) of Type
ScalarType     -> id
                | Bound .. Bound
Bound          -> Sign intconstant
                | id
Sign           -> +
                | -
                |(*@ $\lambda$ @*)
ScalarTypeList -> E F
E              -> ScalarType
F              -> , ScalarType F
                |(*@ $\lambda$ @*)
\end{lstlisting}
%----------------------------------------------------------------------------------------
%   end CRAFTING A COMPILER EXERCISES
%----------------------------------------------------------------------------------------

%----------------------------------------------------------------------------------------
%   REFERENCES
%----------------------------------------------------------------------------------------
% The following two commands are all you need in the initial runs of your .tex file to
% produce the bibliography for the citations in your paper.
%\bibliographystyle{abbrv}
%\bibliography{lab} 
% You must have a proper ".bib" file and remember to run:
% latex bibtex latex latex
% to resolve all references.

\end{document}
